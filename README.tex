% Options for packages loaded elsewhere
\PassOptionsToPackage{unicode}{hyperref}
\PassOptionsToPackage{hyphens}{url}
%
\documentclass[
]{article}
\usepackage{amsmath,amssymb}
\usepackage{iftex}
\ifPDFTeX
  \usepackage[T1]{fontenc}
  \usepackage[utf8]{inputenc}
  \usepackage{textcomp} % provide euro and other symbols
\else % if luatex or xetex
  \usepackage{unicode-math} % this also loads fontspec
  \defaultfontfeatures{Scale=MatchLowercase}
  \defaultfontfeatures[\rmfamily]{Ligatures=TeX,Scale=1}
\fi
\usepackage{lmodern}
\ifPDFTeX\else
  % xetex/luatex font selection
\fi
% Use upquote if available, for straight quotes in verbatim environments
\IfFileExists{upquote.sty}{\usepackage{upquote}}{}
\IfFileExists{microtype.sty}{% use microtype if available
  \usepackage[]{microtype}
  \UseMicrotypeSet[protrusion]{basicmath} % disable protrusion for tt fonts
}{}
\makeatletter
\@ifundefined{KOMAClassName}{% if non-KOMA class
  \IfFileExists{parskip.sty}{%
    \usepackage{parskip}
  }{% else
    \setlength{\parindent}{0pt}
    \setlength{\parskip}{6pt plus 2pt minus 1pt}}
}{% if KOMA class
  \KOMAoptions{parskip=half}}
\makeatother
\usepackage{xcolor}
\usepackage[margin=1in]{geometry}
\usepackage{color}
\usepackage{fancyvrb}
\newcommand{\VerbBar}{|}
\newcommand{\VERB}{\Verb[commandchars=\\\{\}]}
\DefineVerbatimEnvironment{Highlighting}{Verbatim}{commandchars=\\\{\}}
% Add ',fontsize=\small' for more characters per line
\usepackage{framed}
\definecolor{shadecolor}{RGB}{248,248,248}
\newenvironment{Shaded}{\begin{snugshade}}{\end{snugshade}}
\newcommand{\AlertTok}[1]{\textcolor[rgb]{0.94,0.16,0.16}{#1}}
\newcommand{\AnnotationTok}[1]{\textcolor[rgb]{0.56,0.35,0.01}{\textbf{\textit{#1}}}}
\newcommand{\AttributeTok}[1]{\textcolor[rgb]{0.13,0.29,0.53}{#1}}
\newcommand{\BaseNTok}[1]{\textcolor[rgb]{0.00,0.00,0.81}{#1}}
\newcommand{\BuiltInTok}[1]{#1}
\newcommand{\CharTok}[1]{\textcolor[rgb]{0.31,0.60,0.02}{#1}}
\newcommand{\CommentTok}[1]{\textcolor[rgb]{0.56,0.35,0.01}{\textit{#1}}}
\newcommand{\CommentVarTok}[1]{\textcolor[rgb]{0.56,0.35,0.01}{\textbf{\textit{#1}}}}
\newcommand{\ConstantTok}[1]{\textcolor[rgb]{0.56,0.35,0.01}{#1}}
\newcommand{\ControlFlowTok}[1]{\textcolor[rgb]{0.13,0.29,0.53}{\textbf{#1}}}
\newcommand{\DataTypeTok}[1]{\textcolor[rgb]{0.13,0.29,0.53}{#1}}
\newcommand{\DecValTok}[1]{\textcolor[rgb]{0.00,0.00,0.81}{#1}}
\newcommand{\DocumentationTok}[1]{\textcolor[rgb]{0.56,0.35,0.01}{\textbf{\textit{#1}}}}
\newcommand{\ErrorTok}[1]{\textcolor[rgb]{0.64,0.00,0.00}{\textbf{#1}}}
\newcommand{\ExtensionTok}[1]{#1}
\newcommand{\FloatTok}[1]{\textcolor[rgb]{0.00,0.00,0.81}{#1}}
\newcommand{\FunctionTok}[1]{\textcolor[rgb]{0.13,0.29,0.53}{\textbf{#1}}}
\newcommand{\ImportTok}[1]{#1}
\newcommand{\InformationTok}[1]{\textcolor[rgb]{0.56,0.35,0.01}{\textbf{\textit{#1}}}}
\newcommand{\KeywordTok}[1]{\textcolor[rgb]{0.13,0.29,0.53}{\textbf{#1}}}
\newcommand{\NormalTok}[1]{#1}
\newcommand{\OperatorTok}[1]{\textcolor[rgb]{0.81,0.36,0.00}{\textbf{#1}}}
\newcommand{\OtherTok}[1]{\textcolor[rgb]{0.56,0.35,0.01}{#1}}
\newcommand{\PreprocessorTok}[1]{\textcolor[rgb]{0.56,0.35,0.01}{\textit{#1}}}
\newcommand{\RegionMarkerTok}[1]{#1}
\newcommand{\SpecialCharTok}[1]{\textcolor[rgb]{0.81,0.36,0.00}{\textbf{#1}}}
\newcommand{\SpecialStringTok}[1]{\textcolor[rgb]{0.31,0.60,0.02}{#1}}
\newcommand{\StringTok}[1]{\textcolor[rgb]{0.31,0.60,0.02}{#1}}
\newcommand{\VariableTok}[1]{\textcolor[rgb]{0.00,0.00,0.00}{#1}}
\newcommand{\VerbatimStringTok}[1]{\textcolor[rgb]{0.31,0.60,0.02}{#1}}
\newcommand{\WarningTok}[1]{\textcolor[rgb]{0.56,0.35,0.01}{\textbf{\textit{#1}}}}
\usepackage{longtable,booktabs,array}
\usepackage{calc} % for calculating minipage widths
% Correct order of tables after \paragraph or \subparagraph
\usepackage{etoolbox}
\makeatletter
\patchcmd\longtable{\par}{\if@noskipsec\mbox{}\fi\par}{}{}
\makeatother
% Allow footnotes in longtable head/foot
\IfFileExists{footnotehyper.sty}{\usepackage{footnotehyper}}{\usepackage{footnote}}
\makesavenoteenv{longtable}
\usepackage{graphicx}
\makeatletter
\newsavebox\pandoc@box
\newcommand*\pandocbounded[1]{% scales image to fit in text height/width
  \sbox\pandoc@box{#1}%
  \Gscale@div\@tempa{\textheight}{\dimexpr\ht\pandoc@box+\dp\pandoc@box\relax}%
  \Gscale@div\@tempb{\linewidth}{\wd\pandoc@box}%
  \ifdim\@tempb\p@<\@tempa\p@\let\@tempa\@tempb\fi% select the smaller of both
  \ifdim\@tempa\p@<\p@\scalebox{\@tempa}{\usebox\pandoc@box}%
  \else\usebox{\pandoc@box}%
  \fi%
}
% Set default figure placement to htbp
\def\fps@figure{htbp}
\makeatother
\usepackage{svg}
\setlength{\emergencystretch}{3em} % prevent overfull lines
\providecommand{\tightlist}{%
  \setlength{\itemsep}{0pt}\setlength{\parskip}{0pt}}
\setcounter{secnumdepth}{-\maxdimen} % remove section numbering
\usepackage{bookmark}
\IfFileExists{xurl.sty}{\usepackage{xurl}}{} % add URL line breaks if available
\urlstyle{same}
\hypersetup{
  pdftitle={rfcipDemand},
  hidelinks,
  pdfcreator={LaTeX via pandoc}}

\title{rfcipDemand}
\author{}
\date{\vspace{-2.5em}}

\begin{document}
\maketitle

\href{https://www.repostatus.org/\#active}{\pandocbounded{\includesvg[keepaspectratio]{https://www.repostatus.org/badges/latest/active.svg}}}
\href{https://lifecycle.r-lib.org/articles/stages.html\#experimental}{\pandocbounded{\includesvg[keepaspectratio]{https://img.shields.io/badge/lifecycle-experimental-orange.svg}}}
\href{https://github.com/ftsiboe/rfcipDemand/actions/workflows/R-CMD-check.yaml}{\pandocbounded{\includesvg[keepaspectratio]{https://github.com/ftsiboe/rfcipDemand/actions/workflows/R-CMD-check.yaml/badge.svg}}}
\href{https://codecov.io/gh/ftsiboe/rfcipDemand}{\pandocbounded{\includegraphics[keepaspectratio]{https://codecov.io/gh/ftsiboe/rfcipDemand/graph/badge.svg?token=6MKGP8Z5NB}}}
\pandocbounded{\includegraphics[keepaspectratio]{https://img.shields.io/badge/R-\%3E=4.0-blue}}
\href{code_of_conduct.md}{\pandocbounded{\includesvg[keepaspectratio]{https://img.shields.io/badge/Contributor\%20Covenant-2.1-4baaaa.svg}}}
\pandocbounded{\includesvg[keepaspectratio]{https://img.shields.io/badge/License-GPLv3-blue.svg}}

\section{📖 Introduction}\label{introduction}

\texttt{rfcipDemand} provides a reproducible pipeline for analyzing
\textbf{U.S. Federal Crop Insurance Program (FCIP) demand}.

Its functionalities are grounded in the empirical strategies developed
in:

\begin{itemize}
\item
  Tsiboe, F., \& Turner, D. (2023).
  \href{https://doi.org/10.1016/j.foodpol.2023.102505}{\textbf{The crop
  insurance demand response to premium subsidies: Evidence from U.S.
  Agriculture}} Food Policy, 119(3).
\item
  Tsiboe, F., \& Turner, D. (2023).
  \href{https://doi.org/10.1017/age.2023.13}{\textbf{Econometric
  identification of crop insurance participation}} Agricultural and
  Resource Economics Review
\end{itemize}

Specifically, the package helps you:

\begin{itemize}
\tightlist
\item
  🧩 Build county--crop--practice--plan--unit panels from \textbf{USDA
  RMA SOBTPU} and related sources\\
\item
  🔗 Merge \textbf{price and instrument variables}\\
\item
  🌾 Reconcile \textbf{acreage} using FSA and NASS data\\
\item
  📊 Estimate \textbf{FCIP demand systems} with fixed effects and
  two-way cluster-robust covariance\\
\item
  ✅ Produce diagnostics, including robust first-stage strength
\end{itemize}

\textbf{Disclaimer:} This package uses USDA data but is not endorsed by
or affiliated with USDA or the Federal Government.\\
See \url{LICENSE} for terms.

\begin{center}\rule{0.5\linewidth}{0.5pt}\end{center}

\section{📦 Installation}\label{installation}

\begin{Shaded}
\begin{Highlighting}[]
\CommentTok{\# Install from GitHub}
\ControlFlowTok{if}\NormalTok{ (}\SpecialCharTok{!}\FunctionTok{requireNamespace}\NormalTok{(}\StringTok{"devtools"}\NormalTok{, }\AttributeTok{quietly =} \ConstantTok{TRUE}\NormalTok{)) }\FunctionTok{install.packages}\NormalTok{(}\StringTok{"devtools"}\NormalTok{)}
\NormalTok{devtools}\SpecialCharTok{::}\FunctionTok{install\_github}\NormalTok{(}\StringTok{"ftsiboe/rfcipDemand"}\NormalTok{, }\AttributeTok{force =} \ConstantTok{TRUE}\NormalTok{, }\AttributeTok{upgrade =} \StringTok{"never"}\NormalTok{)}
\end{Highlighting}
\end{Shaded}

\begin{center}\rule{0.5\linewidth}{0.5pt}\end{center}

\section{🚀 Quick Start}\label{quick-start}

The two most important functions are:

\begin{itemize}
\tightlist
\item
  \texttt{fcip\_demand\_data\_dispatcher()} → assemble the modeling
  panel\\
\item
  \texttt{fcip\_demand\_sys\_estimate()} → estimate demand equations
\end{itemize}

\subsection{Example 1: Full sample
estimation}\label{example-1-full-sample-estimation}

Model structure aligned to the approach in
\href{https://doi.org/10.1016/j.foodpol.2023.102505}{Tsiboe \& Turner
(2023)}, updated with recent data.\\
\textbf{NOTE:} Results may differ from the published articles due to RMA
data revisions and pipeline improvements in this package.\\
If you need \emph{exact} replication of a paper, please use that study's
dedicated replication package (link to be added).

\textbf{Data}

\begin{Shaded}
\begin{Highlighting}[]
\CommentTok{\# library(rfcipDemand)}
\NormalTok{devtools}\SpecialCharTok{::}\FunctionTok{document}\NormalTok{()}
\CommentTok{\# 1) Identify fields for panel building}
\NormalTok{FCIP\_INSURANCE\_POOL }\OtherTok{\textless{}{-}} \FunctionTok{c}\NormalTok{(}\StringTok{"state\_code"}\NormalTok{,}\StringTok{"county\_code"}\NormalTok{,}\StringTok{"commodity\_code"}\NormalTok{,}\StringTok{"type\_code"}\NormalTok{,}\StringTok{"practice\_code"}\NormalTok{)}

\CommentTok{\# 2) Build data (example years {-} keep short so examples are fast)}
\NormalTok{df }\OtherTok{\textless{}{-}} \FunctionTok{fcip\_demand\_data\_dispatcher}\NormalTok{(}
  \AttributeTok{study\_years =} \DecValTok{2001}\SpecialCharTok{:}\DecValTok{2024}\NormalTok{,}
  \AttributeTok{identifiers =} \FunctionTok{c}\NormalTok{(}\StringTok{"commodity\_year"}\NormalTok{, FCIP\_INSURANCE\_POOL, }\StringTok{"insurance\_plan\_code"}\NormalTok{, }\StringTok{"unit\_structure\_code"}\NormalTok{)}
\NormalTok{)}

\CommentTok{\# 3) Prep variables (toy scaling for demo)}
\NormalTok{data                }\OtherTok{\textless{}{-}} \FunctionTok{as.data.frame}\NormalTok{(df)}
\NormalTok{data}\SpecialCharTok{$}\NormalTok{Gamma          }\OtherTok{\textless{}{-}}\NormalTok{ data}\SpecialCharTok{$}\NormalTok{net\_reporting\_level\_amount }\SpecialCharTok{/} \DecValTok{10000}
\NormalTok{data}\SpecialCharTok{$}\NormalTok{Theta1         }\OtherTok{\textless{}{-}}\NormalTok{ data}\SpecialCharTok{$}\NormalTok{coverage\_level\_percent\_aggregate}
\NormalTok{data}\SpecialCharTok{$}\NormalTok{rate           }\OtherTok{\textless{}{-}}\NormalTok{ data}\SpecialCharTok{$}\NormalTok{premium\_per\_liability}\SpecialCharTok{*}\NormalTok{(}\DecValTok{1}\SpecialCharTok{{-}}\NormalTok{data}\SpecialCharTok{$}\NormalTok{subsidy\_per\_premium)}
\NormalTok{data}\SpecialCharTok{$}\NormalTok{county\_acreage }\OtherTok{\textless{}{-}}\NormalTok{ data}\SpecialCharTok{$}\NormalTok{county\_acreage }\SpecialCharTok{/} \DecValTok{10000}
\NormalTok{data}\SpecialCharTok{$}\NormalTok{rent           }\OtherTok{\textless{}{-}}\NormalTok{ data}\SpecialCharTok{$}\NormalTok{rent }\SpecialCharTok{/} \DecValTok{1000}
\NormalTok{data}\SpecialCharTok{$}\NormalTok{trend          }\OtherTok{\textless{}{-}}\NormalTok{ data}\SpecialCharTok{$}\NormalTok{commodity\_year }\SpecialCharTok{{-}} \FunctionTok{min}\NormalTok{(data}\SpecialCharTok{$}\NormalTok{commodity\_year, }\AttributeTok{na.rm=}\ConstantTok{TRUE}\NormalTok{)}
\NormalTok{data}\SpecialCharTok{$}\NormalTok{FCIP           }\OtherTok{\textless{}{-}} \DecValTok{1}
\NormalTok{data}\SpecialCharTok{$}\NormalTok{tauS0        }\OtherTok{\textless{}{-}}\NormalTok{ data}\SpecialCharTok{$}\NormalTok{tau}\SpecialCharTok{*}\NormalTok{(}\DecValTok{1}\SpecialCharTok{{-}}\NormalTok{((data}\SpecialCharTok{$}\NormalTok{subsidy\_rate\_65}\SpecialCharTok{+}\NormalTok{data}\SpecialCharTok{$}\NormalTok{subsidy\_rate\_75)}\SpecialCharTok{/}\DecValTok{2}\NormalTok{))}

\ControlFlowTok{for}\NormalTok{(i }\ControlFlowTok{in} \FunctionTok{unique}\NormalTok{(data}\SpecialCharTok{$}\NormalTok{commodity\_code))\{ data[,}\FunctionTok{paste0}\NormalTok{(}\StringTok{"Crop\_"}\NormalTok{,i)] }\OtherTok{\textless{}{-}} \FunctionTok{ifelse}\NormalTok{(data}\SpecialCharTok{$}\NormalTok{commodity\_code }\SpecialCharTok{\%in\%}\NormalTok{ i,}\DecValTok{1}\NormalTok{,}\DecValTok{0}\NormalTok{)}\SpecialCharTok{*}\NormalTok{data}\SpecialCharTok{$}\NormalTok{trend \}}
\ControlFlowTok{for}\NormalTok{(i }\ControlFlowTok{in} \FunctionTok{unique}\NormalTok{(data}\SpecialCharTok{$}\NormalTok{commodity\_year))\{ data[,}\FunctionTok{paste0}\NormalTok{(}\StringTok{"year\_"}\NormalTok{,i)] }\OtherTok{\textless{}{-}} \FunctionTok{ifelse}\NormalTok{(data}\SpecialCharTok{$}\NormalTok{commodity\_year }\SpecialCharTok{\%in\%}\NormalTok{ i,}\DecValTok{1}\NormalTok{,}\DecValTok{0}\NormalTok{) \}}
\NormalTok{data }\OtherTok{\textless{}{-}}\NormalTok{ data[}\FunctionTok{names}\NormalTok{(data)[}\SpecialCharTok{!}\FunctionTok{names}\NormalTok{(data) }\SpecialCharTok{\%in\%} \FunctionTok{c}\NormalTok{(}\FunctionTok{paste0}\NormalTok{(}\StringTok{"year\_"}\NormalTok{,}\FunctionTok{max}\NormalTok{(data}\SpecialCharTok{$}\NormalTok{commodity\_year,}\AttributeTok{na.rm=}\NormalTok{T)),}\StringTok{"Crop\_41"}\NormalTok{)]]}
\end{Highlighting}
\end{Shaded}

\textbf{🧮 Estimate the model}

\begin{Shaded}
\begin{Highlighting}[]
\CommentTok{\# 4) Specify the system}

\NormalTok{model }\OtherTok{\textless{}{-}} \FunctionTok{list}\NormalTok{(}
  \AttributeTok{name       =} \StringTok{"demo\_sys"}\NormalTok{,}
  \AttributeTok{FE         =} \ConstantTok{TRUE}\NormalTok{,}
  \AttributeTok{outcome    =} \FunctionTok{c}\NormalTok{(}\StringTok{"Gamma"}\NormalTok{,}\StringTok{"Theta1"}\NormalTok{),}
  \AttributeTok{endogenous =} \StringTok{"rate"}\NormalTok{,}
  \AttributeTok{excluded   =} \StringTok{"tauS0"}\NormalTok{,}
  \AttributeTok{partial    =} \FunctionTok{c}\NormalTok{(}\StringTok{"trend"}\NormalTok{,}\FunctionTok{names}\NormalTok{(data)[}\FunctionTok{grepl}\NormalTok{(}\StringTok{"Crop\_"}\NormalTok{,}\FunctionTok{names}\NormalTok{(data))],}\FunctionTok{names}\NormalTok{(data)[}\FunctionTok{grepl}\NormalTok{(}\StringTok{"year\_"}\NormalTok{,}\FunctionTok{names}\NormalTok{(data))]),}
  \AttributeTok{disag      =} \StringTok{"FCIP"}\NormalTok{,}
  \AttributeTok{included   =} \FunctionTok{c}\NormalTok{(}\StringTok{"county\_acreage"}\NormalTok{,}\StringTok{"price"}\NormalTok{,}\StringTok{"rent"}\NormalTok{)}
\NormalTok{)}

\CommentTok{\# 5) Estimate demand system}
\NormalTok{res }\OtherTok{\textless{}{-}} \FunctionTok{fcip\_demand\_sys\_estimate}\NormalTok{(}\AttributeTok{model =}\NormalTok{ model, }\AttributeTok{data =}\NormalTok{ data)}

\FunctionTok{write.csv}\NormalTok{(res,}\StringTok{"data{-}raw/examples/example1.csv"}\NormalTok{)}
\end{Highlighting}
\end{Shaded}

\textbf{📊 Discussion of Results}

The outputs (see Table 1 below) from
\texttt{fcip\_demand\_sys\_estimate()} are structured objects that
typically include:

\begin{itemize}
\tightlist
\item
  \textbf{System coefficients}: Estimated elasticities of demand with
  respect to premium rates, coverage levels, and control variables.\\
\item
  \textbf{Robust inference}: Standard errors clustered by county and
  year, consistent with best practices in applied demand modeling.\\
\item
  \textbf{First-stage diagnostics}: Strength of excluded instruments
  (e.g., \texttt{tau}), ensuring valid identification of the endogenous
  premium rate.\\
\item
  \textbf{Equation-level summaries}: For multi-equation systems, results
  are returned per outcome (e.g., insured acreage (\texttt{Gamma}) and
  coverage level (\texttt{Theta1})).
\end{itemize}

\textbf{Table 1: Crop Insurance Demand System for US Federal Crop
Insurance Pools (2001/24)}

\begin{Shaded}
\begin{Highlighting}[]
\NormalTok{devtools}\SpecialCharTok{::}\FunctionTok{document}\NormalTok{()}
\CommentTok{\#\textgreater{} i Updating rfcipDemand documentation}
\CommentTok{\#\textgreater{} i Loading rfcipDemand}
\CommentTok{\#\textgreater{} Writing \textquotesingle{}format\_fcip\_demand\_table.Rd\textquotesingle{}}
\FunctionTok{library}\NormalTok{(knitr)}
\NormalTok{example1 }\OtherTok{\textless{}{-}}\NormalTok{ readr}\SpecialCharTok{::}\FunctionTok{read\_csv}\NormalTok{(}\StringTok{"data{-}raw/examples/example1.csv"}\NormalTok{, }\AttributeTok{show\_col\_types =} \ConstantTok{FALSE}\NormalTok{)}
\CommentTok{\#\textgreater{} New names:}
\CommentTok{\#\textgreater{} * \textasciigrave{}\textasciigrave{} {-}\textgreater{} \textasciigrave{}...1\textasciigrave{}}

\CommentTok{\# Variable name mapping}
\NormalTok{var\_labels }\OtherTok{\textless{}{-}} \FunctionTok{c}\NormalTok{(}
  \StringTok{"(Intercept)"}            \OtherTok{=} \StringTok{"(Intercept)"}\NormalTok{,}
  \StringTok{"tilda\_rate"}             \OtherTok{=} \StringTok{"Paid premium rate"}\NormalTok{,}
  \StringTok{"tilda\_county\_acreage"}   \OtherTok{=} \StringTok{"County planted acres"}\NormalTok{,}
  \StringTok{"tilda\_price"}            \OtherTok{=} \StringTok{"Expected crop price"}\NormalTok{,}
  \StringTok{"tilda\_rent"}             \OtherTok{=} \StringTok{"State rental rate for land"}\NormalTok{,}
  \StringTok{"residCov\_11"}            \OtherTok{=} \StringTok{"σ\_aa"}\NormalTok{,}
  \StringTok{"residCov\_22"}            \OtherTok{=} \StringTok{"σ\_θθ"}\NormalTok{,}
  \StringTok{"residCov\_12"}            \OtherTok{=} \StringTok{"σ\_θa"}\NormalTok{,}
  \StringTok{"N"}                      \OtherTok{=} \StringTok{"Number of observations"}\NormalTok{,}
  \StringTok{"NFE"}                    \OtherTok{=} \StringTok{"Number of insurance pools"}\NormalTok{,}
  \StringTok{"JTest"}                  \OtherTok{=} \StringTok{"J{-}test"}\NormalTok{,}
  \StringTok{"FTest"}                  \OtherTok{=} \StringTok{"Weak{-}instrument: F{-}statistics"}
\NormalTok{)}

\NormalTok{final\_tbl }\OtherTok{\textless{}{-}} \FunctionTok{format\_fcip\_demand\_table}\NormalTok{(example1, var\_labels)}

\CommentTok{\# Print table}
\FunctionTok{kable}\NormalTok{(final\_tbl,}
      \AttributeTok{col.names =} \FunctionTok{c}\NormalTok{(}\StringTok{"Variables"}\NormalTok{,}\StringTok{"Estimates"}\NormalTok{),}
      \AttributeTok{format =} \StringTok{"pipe"}\NormalTok{,  }\CommentTok{\# \textless{}{-} ensures compatibility with GitHub markdown}
      \AttributeTok{align =} \FunctionTok{c}\NormalTok{(}\StringTok{"l"}\NormalTok{,}\StringTok{"c"}\NormalTok{))}
\end{Highlighting}
\end{Shaded}

\begin{longtable}[]{@{}lc@{}}
\toprule\noalign{}
Variables & Estimates \\
\midrule\noalign{}
\endhead
\bottomrule\noalign{}
\endlastfoot
Coverage level (ln θ\_it) & \\
(Intercept) & 0.000 (0.002) \\
Paid premium rate & -0.359 (0.423) \\
County planted acres & -0.001 (0.001) \\
Expected crop price & -0.000 (0.001) \\
State rental rate for land & 0.013 (0.519) \\
Insured acres (ln a\_it) & \\
(Intercept) & 0.000 (0.005) \\
Paid premium rate & -0.476 (2.260) \\
County planted acres & 0.024** (0.012) \\
Expected crop price & 0.002 (0.008) \\
State rental rate for land & 0.423 (2.993) \\
Total protection response & \\
Paid premium rate & -0.664 (1.648) \\
County planted acres & 0.023* (0.012) \\
Expected crop price & 0.002 (0.009) \\
State rental rate for land & 0.442 (3.672) \\
Covariance matrix & \\
σ\_aa & 1.349 \\
σ\_θθ & 0.006 \\
σ\_θa & 0.014 \\
Additional statistics & \\
Number of observations & 1917600.000 \\
Number of insurance pools & 1.000 \\
J-test & 0.000 \\
Weak-instrument: F-statistics & 125.044 \\
\end{longtable}

\textbf{Notes:} Crop insurance demand is modeled via a multi-equation
structural model of crop insurance demand at the intensive and extensive
margins measured by coverage level and insured acres. An insurance pool
is defined as the unique combinations of crops (almonds, apples, barley,
blueberries, cabbage, canola, corn, cotton, dry beans, dry peas, flax,
forage production, fresh nectarines, grain sorghum, grapes, millet,
oats, olives, onions, oranges, peaches, peanuts, pears, plums, potatoes,
rice, rye, safflower, soybeans, sugar beets, sugarcane, sunflowers,
sweet corn, tobacco, tomatoes, walnuts, and wheat), county, insurance
unit (optional units {[}OU{]}, basic units {[}BU{]}, or enterprise units
{[}EU{]}), insurance plan, irrigation practice (irrigated,
non-irrigated, or unspecified), and organic practice (organic certified,
organic transition, or unspecified). The data used was constructed by
the authors using primary data from (1) Risk Management Agency's summary
of business files that contain insurance metrics aggregated by county,
crop, crop type, production practice, insurance plan, coverage level,
and insurance unit, actuarial data master, and price addendums, (2) Farm
Service Agency's crop acreage data, and (3) NASS Quick Stats. The
preferred model is ††.\\
Significance levels -- \emph{p\textless0.1, \textbf{p\textless0.05,
}}p\textless0.01. Standard errors in parentheses are clustered by
insurance pool and year.

The results from the preferred specification highlight distinct
responses across the intensive and extensive margins of crop insurance
demand. At the intensive margin (coverage level, ln θ\_it), the
producer-paid premium rate enters with the expected negative sign
(-0.359), implying that a 1\% increase in the premium rate is associated
with a 0.36\% decrease in chosen coverage levels. However, the effect is
statistically insignificant, reflecting the limited responsiveness of
coverage choices to cost signals. Other covariates, including county
planted acres, crop price, and rental rates, are similarly imprecise and
not distinguishable from zero.

At the extensive margin (insured acres, ln a\_it), scale effects
dominate. County planted acres exhibit a positive and statistically
significant coefficient (0.024, p \textless{} 0.05), meaning that a 1\%
increase in planting area raises insured acreage by about 0.024\%. The
premium rate again shows a negative effect (-0.476), suggesting a 1\%
increase in rates reduces insured acreage by nearly 0.48\%, though the
standard error is large and the estimate is not significant.

For the total protection response, county planted acres remain a key
driver (0.023, p \textless{} 0.10), indicating that scale continues to
push overall demand upward by roughly 0.023\% for each 1\% increase in
planted acres. The premium rate reduces total protection (-0.664),
implying that a 1\% increase in paid premiums reduces total protection
demand by about 0.66\%, though again, the estimate is not statistically
precise.

The covariance matrix provides additional insight. The positive
cross-covariance (σ\_θa = 0.014) indicates that unobserved factors
increasing demand for coverage level also raise demand for insured
acres, and vice versa. However, the relationship is asymmetric: the
variance of insured acres (σ\_aa = 1.349) dwarfs that of coverage level
(σ\_θθ = 0.006), suggesting that shocks to acreage drive most of the
variation in joint demand.

Overall, these estimates point to farm size (planted acres) as the most
consistent determinant of insurance demand, while the dampened and
imprecisely estimated response to premium rates underscores how
subsidies mute price sensitivity. The positive covariance between
margins further suggests complementarities in demand, but the dominant
source of variation lies in the extensive margin, highlighting the
central role of scale in shaping crop insurance participation.

\section{📚 Citation}\label{citation}

If you use \texttt{rfcipDemand} in your research, please cite:

\begin{itemize}
\item
  Tsiboe, F., \& Turner, D. (2023).
  \href{https://doi.org/10.1016/j.foodpol.2023.102505}{\textbf{The crop
  insurance demand response to premium subsidies: Evidence from U.S.
  Agriculture}} Food Policy, 119(3).
\item
  Tsiboe, F., \& Turner, D. (2023).
  \href{https://doi.org/10.1017/age.2023.13}{\textbf{Econometric
  identification of crop insurance participation}} Agricultural and
  Resource Economics Review
\end{itemize}

\begin{center}\rule{0.5\linewidth}{0.5pt}\end{center}

\section{🤝 Contributing}\label{contributing}

Contributions, issues, and feature requests are welcome. Please see the
\href{code_of_conduct.md}{Code of Conduct}.

\begin{center}\rule{0.5\linewidth}{0.5pt}\end{center}

\section{📬 Contact}\label{contact}

Questions or collaboration ideas?\\
Email \textbf{Francis Tsiboe} at
\href{mailto:ftsiboe@hotmail.com}{\nolinkurl{ftsiboe@hotmail.com}}.\\
Star the repo ⭐ if you find it useful!

\end{document}
